\documentclass[12pt,letterpaper]{article}
\usepackage[utf8]{inputenc}
\usepackage[spanish]{babel}
\usepackage[T1]{fontenc}
\usepackage[ansinew]{inputenc}
\usepackage{graphicx}

\begin{document}
\begin{portada}\begin{center}
\vspace*{-1in}
\begin{figure}[htb]
\begin{center}
\includegraphics[width=4cm, height=4cm]{escudo.png}
\end{center}
\end{figure}

UNIVERSIDAD DE SONORA\\
\vspace*{0.15in}
DEPARTAMENTO DE CIENCIAS EXACTAS Y NATURALES \\
\vspace*{0.6in}

\begin{Large}
\textbf{Sondeo atmosférico: Del Rio, TX} \\
\end{Large}
\vspace*{0.3in}

\vspace*{0.3in}
\rule{80mm}{0.1mm}\\
\vspace*{0.1in}
\begin{large}
Por Álvarez Sánchez Francisco Eduardo
\end{large}

\begin{large}
Maestro: Dr. Carlos Lizárrga Celaya \\
\end{large}
\vspace*{5cm}
\begin{large}
Hermosillo, Sonora a \today
\end{large}
\end{center}
\thispagestyle{empty}

\end{titlepage}
%%%%%%%%%%%%%%%%%%%%%%%%%%%%%%%%%%%%%%%%%%%%%%%%%%%%%%%%%%%%%%%%%%%%%
\tableofcontents
\cleardoublepage
%\addcontentsline{toc}{section}{1.Introducción} % para que aparezca en el indice de contenidos

%\listoffigures En el siguiente texto se presenta el sondeo de 

%\addcontentsline{toc}{section}{Resúmen}

%\cleardoublepage
%\addcontentsline{toc}{chapter}{Lista de tablas} % para que aparezca en el indice de contenidos
%\listoftables % indice de tablas


%%%%%%%%%%%%%%%%%%%%%%%%%%%%%%%%%%%%%%%%%%%%%%%%%%%%%%%%%%%%%%%%%%%%%
\section{Introducción}

En el siguiente documento utilizaremos datos recolectados por sondas atmosféricas que se lanzan varias veces al día, de la red de sondeos de Norteamérica.\\
Los datos son recolectados y reportados por diversas estaciones y organizados por la Universidad de Wyoming. En este documento en específico nos enfocaremos en los datos obtenidos en Del Río, Texas en Estados Unidos. Verificaremos los datos y graficaremos algunos resultados obtenidos.

\section{Conociendo un poco de Del Río, TX}
Del Río es una ciudad sede del condado de Val Verde, Texas. Se encuentra en la parte sur de Texas. El Área Metropolitana Del Río-Ciudad Acuña (DR-CA) es uno de las seis áreas metropolitanas bi-nacionales a lo largo de la frontera de Estados Unidos con México. La ciudad de Del Río está situada en el estado estadounidense de Texas al lado norte del río Grande y Ciudad Acuña que se encuentra en el estado mexicano de Coahuila al sur del río.
\subsection{Geografía}
Cuenta con una superficie total de 52.13 kilómetros cuadrados y se encuentra a una altitud de 315 metro sobre el nivel del mar. \\
Cuenta con una población de 35,591 habitantes y con una densidad de 680,35 habitantes por cada kilometro cuadrado. Del Río se encuentra ubicada en las coordenadas 29°22'51''N 100°52'51'' O

\begin{center}
\includegraphics[width=5cm, height=5cm]{mapa.png}
\end{center}

\section{¿Que son los sondeos?}
Un Sondeo es una herramienta de observación ejecutada por aquellos que deseen tener claro cuál es el panorama en una determinada cuestión, un sondeo es por definición un proceso destinado a la búsqueda de un resultado estadístico el cual da la idea de lo que se quiere aplicar en la zona en la que se realizó dicho procedimiento. La palabra proviene de un modismo, una sonda es un objeto, manipulado de manera remota el cual es colocado en un lugar y este se dispone a realizar una búsqueda minuciosa de algún elemento en específico al que este configurado para reconocer.

\subsection{Sondeos atmosféricos}
Específicamente hablando de los sondeos atmosféricos, estos buscan obtener información acerca de muchas propiedades que posee nuestra atmósfera. Esto va desde la temperatura hasta las presiones experimentadas a diferentes alturas. También estos sondeos pueden recolectar información como la humedad relativa, la cantidad de vapor de agua en diferentes zonas de la atmósfera, entre cosas.

Estos sondeos se realizan gracias a una herramienta llama globos atmosféricos o globos meteorológicos. Estos cargan con medidores de diferentes tipos, que son los que mandan la información hacia tierra para poder organizarla e interpretarla de alguna manera.
\subsection{Interpretación de los sondeos atmosféricos}
Para interpretar los datos obtenidos de los globos se necesitan herramientas como procesadores de datos o programas que ayuden a interpretarlo en forma de graficas o tablas. 


\section{Sondeo atmosferico: Del Río, Texas}
En el siguiente apartado se mostraran los resultados de los sondeos de la ciudad ya menciona. Se interpretaran los datos y mostraran en forma de gráficas. También se hará un conteo de las veces que se realizaron los procesos de sondeo en esta localidad.


\subsection{Presión contra altura}
En la gráfica de Presión contra Altura se muestra como es el cambio de presión con respecto a la altura en la que se encuentra situado el medidor. Como sabemos por lo estudiado en Fluidos y fenómenos térmicos, dependiendo de los compuestos que conforman a cada capa de la atmósfera, así como también la cantidad de capas de atmósfera hay sobre uno al estar mas abajo.\\\\

\begin{figure}[htb]
\begin{center}
\includegraphics[width=12cm, height=12cm]{PresionvsAltura.png}
\end{center}
\end{figure}

\subsection{Temperatura contra altura}
En esta última gráfica, temperatura contra altura, pudimos observar como es la variación de la temperatura respecto a que capa se encuentra el medidor. Podemos observar que entre los 10000 y 12000 metros la temperatura se mantiene, despues vuelve a bajar hasta aproximadamente los -75 C y de ahí vuelve a subir hasta llegar a los -40 C a una altura de 30000 metros de altura. 
\begin{figure}[htb]
\begin{center}
\includegraphics[width=12cm, height=12cm]{TemperaturavsAltura.png}
\end{center}
\end{figure}
\\\\


\subsection{Tabla de mediciones}
\begin{table}[htbp]
\begin{center}
\begin{tabular}{|l|l|l|}
\hline \hline
mes & 00Z & 12Z  \\
\hline \hline
Enero & 31 & 31 \\ \hline
Febrero & 28 & 28\\ \hline
Marzo & 31 & 31\\ \hline
Abril & 30 & 30 \\ \hline
Mayo & 31 & 31 \\ \hline
Junio & 30 & 29 \\ \hline
Julio & 31 & 31 \\ \hline
Agosto & 31 & 31 \\ \hline
Septiembre & 30 & 28 \\ \hline
Octubre & 31 & 31 \\ \hline
Noviembre & 30 & 30 \\ \hline
Diciembre & 31 & 31\\ \hline
\end{tabular}
\caption{observaciones realizadas a las 00Z y 12Z horas.}
\label{tabla:sencilla}
\end{center}
\end{table}


Como podemos observar en la tabla, las mediciones la mayoría de los meses se cumplieron diariamente. Hubo falta de muy pocas mediciones de algunos meses.


\section{Script utilizado}

\begin{verbatim}
# Descarga por mes. Cambiar año de consulta. Ajustar el numero de estacion.
#!/bin/bash
# Despues de editar: chmod 755 script1.sh
# Para ejecutar: ./script1.sh
IFS=":"
LISTM31="01:03:05:07:08:10:12"
#LISTM31="01:03:05:07"
LISTM30="04:06:09:11"
#LISTM30="04:06"
LISTM28="02"
# Script para bajar info por mes. Año 2016, dentro del URL:  YEAR=2015
# Months 31 days
for i in $LISTM31 ; do
    /usr/bin/wget "http://weather.uwyo.edu/cgi-bin/sounding?region=naconf&TYPE=TEXT%3ALIST&YEAR=2016&MONTH=$i&FROM=0100&TO=3112&STNM=72261"
       /bin/sleep 5
done
# Months 30 days
for i in $LISTM30 ; do
    /usr/bin/wget "http://weather.uwyo.edu/cgi-bin/sounding?region=naconf&TYPE=TEXT%3ALIST&YEAR=2016&MONTH=$i&FROM=0100&TO=3012&STNM=72261"
       /bin/sleep 5
done
# Feb. 28 days
for i in $LISTM28 ; do
    /usr/bin/wget "http://weather.uwyo.edu/cgi-bin/sounding?region=naconf&TYPE=TEXT%3ALIST&YEAR=2016&MONTH=$i&FROM=0100&TO=2812&STNM=72261"
       /bin/sleep 5
done
\end{verbatim}

El script anterior fue utilizado para descargar los datos desde la página que fue utilizada para obtener los datos. Este script sirvió para ordenar adquirir los datos para después poderlos utilizar desde un texto de emacs y así poder filtrar algunos datos y graficarlos.

\section{Conclusión}
Gracias a este reporte pudimos tener una idea clara acerca de como es que se conoce todo lo que sabemos de la atmósfera. Vimos que es muy fácil trabajar con los datos obtenidos de algunos medidores y como podemos representarlos de diferentes formas.\
Aprendimos las diferentes maneras de filtrar información para solo quedarnos con la necesaria o con la que utilizaremos.\
Pudimos comprobar algunas cosas que ya se habían estudiado en la actividad anterior, pudimos ver como se cumplían las propiedades para nuestros datos. 
\\\\


\section{Bibliografía}

Carlos Lizárraga Celaya . (1 de febrero de 2017). Iniciándose con el Editor de texto Gnu Emacs. 8 de Febrero de 2017, de Universidad de Sonora Sitio web: http://computacional1.pbworks.com/w/page/114991294/Actividad\%202\%2(2017-1)



Universidad de Wyoming,Department of Atmospheric Science, http://weather.uwyo.edu/upperair/sounding.html \today

\end{document}
