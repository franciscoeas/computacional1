\documentclass[12pt,letterpaper]{article}
\usepackage[utf8]{inputenc}
\usepackage[spanish]{babel}
\usepackage[T1]{fontenc}
\usepackage[ansinew]{inputenc}
\usepackage{graphicx}
\begin{document}
\begin{portada}\begin{center}
\vspace*{-1in}
\begin{figure}[htb]
\begin{center}
\includegraphics[width=4cm, height=4cm]{escudo.png}
\end{center}
\end{figure}

UNIVERSIDAD DE SONORA\\
\vspace*{0.15in}
DEPARTAMENTO DE CIENCIAS EXACTAS Y NATURALES \\
\vspace*{0.6in}

\begin{Large}
\textbf{Introducción a LATEX} \\
\end{Large}
\vspace*{0.3in}

\vspace*{0.3in}
\rule{80mm}{0.1mm}\\
\vspace*{0.1in}
\begin{large}
Por Alvarez Sanchez Franscisco Eduardo
\end{large}
\vspace*{0.1in}\\\\
\begin{large}
Maestro: Dr. Carlos Lizárrga Celaya \\
\end{large}
\vspace*{5cm}
\begin{large}
Hermosillo, Sonora a 30 de Enero de 2017
\end{large}
\end{center}

\end{titlepage}

\\\\
En el siguiente documento compartiré ideas de y refrexiones acerca del uso de LATEX para la elaboración de reportes de trabajo o articulos.

La primera impresión que deja Latex cuando inicias por primera vez un trabajo es un poco caótica, esto gracias a que puede parecer muy complicado entender ese modo de lenguaje, pero como todo lenguaje de programación es importante desde un principio conocer los comandos necesarios para utilizar las funciones más básicas.\\
Por lo tanto Latex al principio deja una impresión de ser complicado, sin embargo después se vuelve una necesidad el trabajar con el ya que cuenta con muchas facilidades al escribir textos científicos

Los aspectos que mas gustan de este lenguaje es su practicidad y su limpieza puesto que aparenta ser algo complicado y una vez que tienes práctica con el se vuelve sencillo y una forma de trabajar ordenada. 

Algo difícil de realizar en Latex que yo no pude lograr fue el agregar gráficos de formas para realizar una portada con mayor formalidad y presentación.

En comparación con otros editores Latex pierde en la cuestión de que si se maneja en el ámbito general no es muy popular y la gente dejara de usarlo porque para utilidades simples no es muy amigable. Sin embargo se vuelve una buena herramienta cuando se escribe de manera científica, cosa que en los editores de texto es muy complicada y tediosa.

Lo que mas llamo la atención de hacer este trabajo en Latex fue que aprendí bastante sobre otro lenguaje que en un futuro quizá sea el mas usado por mí. Aprendí algunas bases y funcionalidades de Latex, así como identificar cuando es mas fácil trabajar con este editor. 

Creo que a mi actividad lo que le falto fue nada más que estética ya que es bastante sobrio, sin embargo considero que la información es completa.

Como comentario adicional me gustaría agregar lo siguiente:
Latex es una ventaja para nosotros los físicos o futuros físicos ya que todo lo que se tornaba complicado o tedioso en otras plataformas con Latex te toma muy poco tiempo. Me hubiése gustado que hubiera un curso formal del uso de Latex ya que así podríamos aprovechar el cien por ciento de las ventajas de este editor. 

\end{document}


